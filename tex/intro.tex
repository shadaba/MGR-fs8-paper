\section{Introduction}
\label{sec:intro}

The Genral Realtivity (GR) is by far most successful theory of gravity to describe the evolution of universe\citep{Peebles1980,Davis83}. The prediction of GR for the evolution of growth of structure is getting precise as we are improving on the precision of cosmological parameters with improved CMB measurement. The GR successfully explains almost all the observation except the physics of mysterious accelerated expansion of the universe and dark matter. The physical understanding of accelerated expansion of the universe is one of the most important  challenge in front of modern physics \citep{Riess1998, Perlmutter1999}. The theoretical explanation of accelerated expansion in GR comes as cosmological constant $\Lambda$ \citep{Einstein1915}. The $\Lambda$ is in good agreement with baryon acoustic oscilation (BAO) \citep{Eis2005, Cole2005, Hutsi2006, Kazin2010, Percival2010, Reid2010, Eric2014, Anderson2014} , supernovae \citep{Suzuki2012, Conley2011}, and CMB \citep{Planck2013, Wmap2013} observations. But, The vacuum energy density predicted by quantum field theory \citep{Weinberg1989} is many order of magnitude higher than observed $\Lambda$.  Some alternative to $\Lambda$ is proposed, which can be divided into two classes first modified gravity \citep{Silvestri2009, Clifton2012} and second dynamical dark energy \citep{Copeland2006}. 

A major challenge in front of modern cosmology is to do precision test of standard model of cosmology ($\Lambda$CDM-GR) and identify areas of tension. Any departure from standard $\Lambda$CDM is likely to be small and challenging to detect according to current Observations. In this 100th anniversary of General Theory of Relativity, we are in a unique position to test the theory to unprecedented precision with the modern observational probe. The deviation from $\Lambda$CDM-GR can be evaluated in mainly two different ways. One is by looking at extension of $\Lambda$CDM-GR by introducing additional parameters for dark energy equation of state and curvature of the universe. The other method is by replacing GR with modified gravity models. The three basic assumptions of $\Lambda$CDM-GR which are popularly tested are the curvature of the universe, nature of dark energy and law of gravitational interaction at large scale. The curvature of universe is measured by relaxing the assumption that spatial curvature is zero and allow $\Omega_K$  parameter to be free. The $\Lambda$CDM-GR assumes that dark energy equation of state is constant ($w=-1$). The popular approach to test the deviation of dark energy equation of state is by allowing $w$ to be a free parameter. Another model to test the time-dependence of dark energy by parametrizing $w$ in terms of $w_0$ and $w_a$  using $w=w_0 + w_a \frac{z}{1+z}$. The nature of gravity in $\Lambda$CDM-GR is tested by replacing the GR with various modified gravity models. One of the important probe which gets affected by these modification is the growth rate. The modified gravity models generally predicts different growth rates, where as different dark energy model predict different evolution of growth rate with redshift.

The modern galaxy redshift survey have successfully measured the growth rate using Redshift Space Distortion (RSD). The distortion produced in the galaxy auto-correlation function due to the peculiar velocity component of the galaxy redshift is known as redshift space distortion. RSD on linear scales reflects the distribution of matter over-density and peculiar velocity of galaxies. Recent galaxy redshift surveys have provided the measurement of growth rate ($f\sigma_8(z)$) upto redshift of $0.8$ , where $f$ is logarithmic derivative of growth factor and $\sigma_8$ is the rms amplitude of matter fluctuation in a sphere of radius 8 h$^{-1}$Mpc. In this paper, we will test all the three assumptions of $\Lambda$CDM-GR listed above using the Planck CMB measurement \citep{Planck2013} and latest RSD measurement  from BOSS CMASS \citep{Alam2015},  SDSS LRG \citep{SDSSLRG2012}, 6dFGRS\citep{6dFGRS}, 2dFGRS  \citep{2dFGRS}, WiggleZ \citep{Blake2011} and VIMOS Public Extragalactic Redshift Survey (VIPERS,\cite{Vipers})